\documentclass[12pt]{article}

\usepackage{sbc-template}
\usepackage[ruled,linesnumbered,lined,commentsnumbered]{algorithm2e}
\usepackage{graphicx,url}
\graphicspath{ {img/} }

\usepackage[brazil]{babel}   
%\usepackage[latin1]{inputenc}  
\usepackage[utf8]{inputenc}  
% UTF-8 encoding is recommended by ShareLaTex
\usepackage{verbatim}
\usepackage{listings}
\usepackage{xcolor}
%\usepackage[table,xcdraw]{xcolor}

\definecolor{verde}{rgb}{0,0.5,0}

%para customizar o código (ver https://en.wikibooks.org/wiki/LaTeX/Source_Code_Listings)
\lstset{language=C, %defina a linguagem usada no trabalho
              belowcaptionskip=1\baselineskip,
                breaklines=true,
                frame=false,
                xleftmargin=\parindent,
                showstringspaces=false,
                basicstyle=\footnotesize\ttfamily,
                keywordstyle=\bfseries\color{green!40!black},
                commentstyle=\itshape\color{purple!40!black},
                identifierstyle=\color{blue},
                stringstyle=\color{orange},
                numbers=left,
            }

\sloppy

\title{Documentação TAD AEDS1}

\author{André Garcia Vieira\inst{1}}


\address{Departamento de Ciência da Computação (DCC) -- UFMG\\
  Av. Pres. Antônio Carlos, 6627 -- Pampulha, Belo Horizonte -- MG, 31270-901
\email{papitow@msn.com}
}

\begin{document} 

\maketitle

\begin{resumo}
Foi implementado um programa que recebe como entrada um arquivo contendo pontos, linhas e polígonos
\end{resumo}

\section{Especificação do Problema}

\textit{(Descreva exatamente como o jogo funciona a sua implementação e como jogar)}

Lorem ipsum dolor sit amet, consectetur adipiscing elit, sed do eiusmod tempor incididunt ut labore et dolore magna aliqua. Ut enim ad minim veniam, quis nostrud exercitation ullamco laboris nisi ut aliquip ex ea commodo consequat. Duis aute irure dolor in reprehenderit in voluptate velit esse cillum dolore eu fugiat nulla pariatur. Excepteur sint occaecat cupidatat non proident, sunt in culpa qui officia deserunt mollit anim id est laborum.

\section{Algoritmos e Linguagem utilizada}

\textit{(Descreva o porque utilizou tal linguagem e decisões tomadas durante a implementação)}

Lorem ipsum dolor sit amet, consectetur adipiscing elit, sed do eiusmod tempor incididunt ut labore et dolore magna aliqua. Ut enim ad minim veniam, quis nostrud exercitation ullamco laboris nisi ut aliquip ex ea commodo consequat. Duis aute irure dolor in reprehenderit in voluptate velit esse cillum dolore eu fugiat nulla pariatur. Excepteur sint occaecat cupidatat non proident, sunt in culpa qui officia deserunt mollit anim id est laborum.

\section{Arquitetura de Comunicação}

\textit{(Como funciona a comunicação entre seu cliente e o servidor, destacando as principais funções, tratamento do ipv6 e ipv4)}

Lorem ipsum dolor sit amet, consectetur adipiscing elit, sed do eiusmod tempor incididunt ut labore et dolore magna aliqua. Ut enim ad minim veniam, quis nostrud exercitation ullamco laboris nisi ut aliquip ex ea commodo consequat. Duis aute irure dolor in reprehenderit in voluptate velit esse cillum dolore eu fugiat nulla pariatur. Excepteur sint occaecat cupidatat non proident, sunt in culpa qui officia deserunt mollit anim id est laborum.

\section{Execuções e Testes}

\textit{(Print dos testes usando IPv4,IPv6, tela do ganhador, tela do perdedor, como executou os programas, os argumentos passados, etc)}

Lorem ipsum dolor sit amet, consectetur adipiscing elit, sed do eiusmod tempor incididunt ut labore et dolore magna aliqua. Ut enim ad minim veniam, quis nostrud exercitation ullamco laboris nisi ut aliquip ex ea commodo consequat. Duis aute irure dolor in reprehenderit in voluptate velit esse cillum dolore eu fugiat nulla pariatur. Excepteur sint occaecat cupidatat non proident, sunt in culpa qui officia deserunt mollit anim id est laborum.

\section{Material Bibliográfico}
\textit{(Sites que utilizou como ajuda)}
\end{document}
